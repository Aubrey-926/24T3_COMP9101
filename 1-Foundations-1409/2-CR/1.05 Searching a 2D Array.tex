\documentclass[12pt]{article}
\usepackage{algos-tasks}

\begin{document}
\task[regular]{Searching a 2D Array}

\begin{question}
In this question, a 2D array is said to be \emph{sorted} if each row and each column is strictly increasing. The following is an example of a sorted 2D array.
\[\begin{bmatrix}
    2 & 3 & 6 & 8 \\
    4 & 5 & 7 & 10 \\
    6 & 8 & 9 & 13 \\
    9 & 11 & 12 & 15
\end{bmatrix}\]

We want to design an $O(n)$ algorithm that determines whether a given value $x$ is in a sorted 2D array $A$ with $n$ rows and columns. The $O(n)$ algorithm must apply to any sorted 2D array and \textbf{not} the above example sorted 2D array only.

\begin{enumerate}
    \item Suppose we have looked at the top right entry of the array, $A[1][n]$. If $A[1][n] < x$, which parts of the array is it possible for $x$ to appear in? Which parts is it \textit{impossible} for $x$ to appear in? Justify your answer.
    \item After this test, what are the dimensions of the remaining search range?

    \item We define an \emph{invariant} as a property of a mathematical object (i.e. an array) that remains unchanged after certain operations. Given (a) and (b), what \emph{invariant} can be maintained each time we compare $x$ with some entry $A[i][j]$ given that we eliminate the parts of $A$ where $x$ cannot be? 

    \textbf{Hint:} In (a), we considered the case where $A[1][n] < x$. You should also consider the cases where $A[1][n] > x$ and $A[1][n] = x$. In each case, what can we say about where $x$ could be in the array?
    
    \item Hence, design an $O(n)$ algorithm that determines whether $x$ appear in a sorted array $A$.

    Using your invariant, justify that your algorithm correctly decides whether or not $x$ is in the array. Start with an outline of your proof structure, and then write your justification. Your algorithm needs to answer `yes' \textit{if and only if} the value $x$ appears in $A$ - this should inform the structure of your justification. You should refer to Tutorial Sheet 1 or the justification examples document in the task resources for some ideas on how to approach your proof.

    Justify that your algorithm runs in $O(n)$ time.

\end{enumerate}
\end{question}
\newpage
\begin{rubric}
\begin{enumerate}
    \item Describe the part of the array that could contain $x$ if $A[1][n] < x$.
        
    Explain why this is the case.
    
    A diagram is optional.
    
    Expected length: two or three sentences.
    \item Identify the dimensions of the remaining array after one comparison, by applying the method of part (a).
    
    Expected length: one sentence.
    \item  Describe an invariant about the remaining search range and the target value $x$.
    
    Expected length: one sentence.
    \item Describe an algorithm that determines whether $A$ contains $x$.
    
    Your algorithm must be written in English, not code or pseudocode, and not transcribed from code or pseudocode.
    
    Justify that your algorithm produces the correct answer.
    
    Justify that your algorithm runs in $O(n)$ time.
\end{enumerate}
\end{rubric}

\begin{solution}
\end{solution}

\begin{attribution}
\end{attribution}

\end{document}