\documentclass[12pt]{article}
\usepackage{algos-tasks}

\begin{document}
\task[regular]{Linked List Rotation}

\begin{question}
Rotating a list right means moving each element along one place, with the previous last value becoming the new head. For example, if we rotate $[1, 2, 3, 4]$ right, we would get $[4, 1, 2, 3]$.

We can represent a non-empty linked list as a pair of 2 things: a value (the `head'), and a smaller list containing all the values that come after the head (the `tail'). So the list $[1, 2, 3, 4]$ is the pair $(h, T)$, where $h = 1$ and $T$ is the list $(2, (3, (4))$. This is similar to a typical linked list in a programming language like C, where $h$ is equivalent to \verb|node.value| and $T$ is equivalent to \verb|node.next|.

This algorithm rotates a linked list $L = (h, T)$ right, where $h$ is the first value and $T$ is the remainder of the list.
\begin{quote}
    If $L$ or $T$ is the empty list, then just return $L$.
    
    Otherwise, recursively rotate $T$, the result of which we denote $(h', T')$. Note that $T'$ may be empty.
    
    Output $(h', (h, T'))$.
\end{quote}

For example, to rotate $L = (1, (2, (3, (4))))$  (with $h=1$), we would start by recursively rotating $T = (2, (3, (4)))$. Rotating $T$ produces $(4, (2, (3)))$, which gives $h' = 4$ and $T' = (2, (3))$, so we output the rotation of $L$ as $(4, (1, (2, (3))))$.

By using induction on the size of the list, show that this algorithm correctly rotates a linked list right.

\textbf{Hint:} Where was $h'$ before rotating $T$?
\end{question}

\begin{rubric}

Show by induction that the rotation algorithm is correct.

Expected length: 2 or 3 short paragraphs.
\end{rubric}

\begin{solution}
\end{solution}

\begin{attribution}
\end{attribution}

\end{document}