\documentclass[12pt]{article}
\usepackage{algos-tasks}

\begin{document}
\task[regular]{Line Segment Intersection}

\begin{question}
You are given a set $H$ of horizontal line segments and a set $V$ of vertical line segments such that $\lvert H \rvert + \lvert V \rvert = n$. Each horizontal line segment is specified by two $x$-coordinates (left and right) and one $y$-coordinate: $\{\ell_i, r_i, y_i\}$ and each vertical line segment is specified by two $y$-coordinates (down and up) and one $x$-coordinate: $\{d_i, u_i, x_i\}$.

{\em In all three subparts, you may assume that all of the $x$ and $y$ coordinates are distinct; in other words, no pair of horizontal lines intersect and no pair of vertical lines intersect.}

\begin{enumerate}[label = (\alph*)]
    \item Firstly, suppose that each vertical line segment is unbounded from above and below. In other words, each vertical line segment stretches from $-\infty$ to $+\infty$ in its $y$-coordinate.

    Design and analyse an $O(n \log n)$ algorithm to return the number of intersections between line segments in $H$ and {\em lines} in $V$.

    For example, the following diagram illustrates three horizontal line segments and four vertical lines, which give rise to six intersection points.
    \begin{center}
        \begin{tikzpicture}
            \begin{scope}[every path/.style={thick, red, latex-latex}]
                \draw (0.5, -1.5) -- (0.5, 1.5);
                \draw (3, -1.5) -- (3, 1.5);
                \draw (1.5, -1.5) -- (1.5, 1.5);
                \draw (2, -1.5) -- (2, 1.5);
            \end{scope}

            \begin{scope}[every node/.style={fill = blue, circle, inner sep = 0pt, minimum size = 4pt}]
                \node (a) {};
                \node[right = 2cm of a] (b) {};
                \node[below right = 1cm of a] (c) {};
                \node[right = 0.75cm of c] (d) {};
                \node[above right = 1cm of a] (e) {};
                \node[right = 1.75cm of e] (f) {};
            \end{scope}

            \begin{scope}[every path/.style={thick, blue}]
                \draw (a) -- (b);
                \draw (c) -- (d);
                \draw (e) -- (f);
            \end{scope}
        \end{tikzpicture}
    \end{center}

    \textbf{Hint:} sort the $x$-coordinates and sweep from left to right.

    \item Now, suppose that each vertical line is bounded from above. In other words, each vertical line has a top endpoint but each vertical line stretches downwards to $-\infty$. We call these {\em rays}, and each ray can be specified by one $x$-coordinate (up) and one $y$-coordinate $\{u_i, y_i\}$.
    
    Design and analyse an $O(n\log^2n)$ algorithm that returns the number of intersections between line segments in $H$ and {\em rays} in $V$.

    \textbf{Note:} the recurrence $T(n) = 2\,T(\frac{n}{2}) \,+ \,O(n \log n)$ has asymptotic solution $T(n) = \Theta(n\log^2n)$; this is an extension of case 2 of the Master Theorem.

    For example, the following diagram illustrates three horizontal line segments and four vertical rays, which give rise to three intersection points.
    \begin{center}
        \begin{tikzpicture}
            \begin{scope}[every node/.style={fill = red, circle, inner sep = 0pt, minimum size = 4pt}]
                \node at (0.5, 1.5) {};
                \node at (1.5, -0.25) {};
                \node at (2, 0.65) {};
                \node at (3, 1.25) {};
            \end{scope}
            
            \begin{scope}[every path/.style={thick, red, latex-}]
                \draw (0.5, -1.5) -- (0.5, 1.5);
                \draw (3, -1.5) -- (3, 1.25);
                \draw (1.5, -1.5) -- (1.5, -0.25);
                \draw (2, -1.5) -- (2, 0.65);
            \end{scope}

            \begin{scope}[every node/.style={fill = blue, circle, inner sep = 0pt, minimum size = 4pt}]
                \node (a) {};
                \node[right = 2cm of a] (b) {};
                \node[below right = 1cm of a] (c) {};
                \node[right = 0.75cm of c] (d) {};
                \node[above right = 1cm of a] (e) {};
                \node[right = 1.75cm of e] (f) {};
            \end{scope}

            \begin{scope}[every path/.style={thick,blue}]
                \draw (a) -- (b);
                \draw (c) -- (d);
                \draw (e) -- (f);
            \end{scope}
        \end{tikzpicture}
    \end{center}

    \textbf{Hint:} divide the problem into \emph{four} subproblems, each of which either:
    \begin{itemize}
        \item contributes no intersection points,
        \item is equivalent to the problem in (a), or
        \item is a smaller instance of the problem in (b), which we can solve recursively.
    \end{itemize}

    \item We now revisit the original problem. Suppose that each vertical ray is bounded from below. In other words, each vertical ray has a top and bottom endpoint. Each of these vertical line segments can be specified by two $y$-coordinates (down and up) and one $x$-coordinate: $\{d_i, u_i, x_i\}$.
    
    Design and analyse an $O(n\log^2n)$ algorithm that returns the number of intersections between line segments in $H$ and line segments in $V$.

    For example, the following diagram illustrates three horizontal line segments and four vertical line segments, which give rise to three intersection points.
    \begin{center}
        \begin{tikzpicture}
            \begin{scope}[every node/.style={fill = red, circle, inner sep = 0pt, minimum size = 4pt}]
                \node at (0.5, -1.5) {};
                \node at (0.5, 1.5) {};
                \node at (1.5, -0.25) {};
                \node at (1.5, -1.25) {};
                \node at (2, 0.65) {};
                \node at (2, -0.5) {};
                \node at (3, 1.25) {};
                \node at (3, 0) {};
            \end{scope}

            \begin{scope}[every path/.style={thick, red}]
                \draw (0.5, -1.5) -- (0.5, 1.5);
                \draw (3, 0) -- (3, 1.25);
                \draw (1.5, -1.25) -- (1.5, -0.25);
                \draw (2, -0.5) -- (2, 0.65);
            \end{scope}

            \begin{scope}[every node/.style={fill = blue, circle, inner sep = 0pt, minimum size = 4pt}]
                \node (a) {};
                \node[right = 2cm of a] (b) {};
                \node[below right = 1cm of a] (c) {};
                \node[right = 0.75cm of c] (d) {};
                \node[above right = 1cm of a] (e) {};
                \node[right = 1.75cm of e] (f) {};
            \end{scope}

            \begin{scope}[every path/.style={thick, blue}]
                \draw (a) -- (b);
                \draw (c) -- (d);
                \draw (e) -- (f);
            \end{scope}
        \end{tikzpicture}
    \end{center}

    \textbf{Hint:} run the algorithm from (b) twice in a smart way.

    \item \textbf{(optional challenge)} Can you solve the same problem in $O(n\log n)$ time?

    \textbf{Hint:} start by reducing the complexity of part (b) to $O(n \log n)$ using precomputation.
\end{enumerate}
\end{question}

\begin{rubric}
\begin{itemize}
    \item Your solution should clearly outline which subpart you're answering, especially if you are doing any of the optional parts.

    \item You can refer to the solution of previous parts, whether you did them or not.

    \item You should argue the correctness of the algorithm and its time complexity.

    \item Course materials such as lectures and tutorials don't require attribution, but making explicit reference to similar ideas is still helpful to show your understanding.
\end{itemize}

Expected length: around two pages.
\end{rubric}

\begin{solution}
\end{solution}

\begin{attribution}
\end{attribution}
\end{document}